\section{Background}
Last semester, we designed and implemented Felix: a system which measures
network traffic using NetKAT. Felix's query language and query compilation
design were complete, and a paper on Felix was accepted to SOSR. However,
both Felix and the SOSR submission were imperfect. First, Felix's query
compiler was slow for large inputs and would often crash on \emph{very} large
inputs. Second, the Felix compiler was not integrated with Haoxian's runtime
which made it onerous to run Felix end-to-end. Finally, our SOSR submission was
written hurriedly and submitted last minute\footnote{literally!}, leaving the
paper a bit incohesive.

\section{Getting Felix Camera-Ready}
The bulk of this semester was spent fixing these imperfections. We optimized
the Felix compiler, integrated the compiler with the runtime, conducted new
benchmarks and case studies, revised the paper, and ultimately presented Felix
at SOSR!

\paragraph{Optimization}
% michael: hashconsing
% michael: CPS to prevent crashing
% rene:    splitting up predicates and compiling them separately

\paragraph{Integration}
% brandon + jake: integrating with Haoxian

\paragraph{Experimentation}
% ?: talk about all the benchmarks and integration and stuff we did

\paragraph{Revision}
% anyone: talk about how we rewrote the paper, but mention it was mostly nate

\paragraph{Presentation}
% michael

\section{Budding Ideas}
% ???; whoever was at these meetings:
%   - detecting cycles and black holes
%   - checking counts with minimal instrumentation

\section{Integrating Code}
% michael
\paragaph{What's Done}
\paragaph{What's Left}
